\documentclass[]{article}
\usepackage{lmodern}
\usepackage{amssymb,amsmath}
\usepackage{ifxetex,ifluatex}
\usepackage{fixltx2e} % provides \textsubscript
\ifnum 0\ifxetex 1\fi\ifluatex 1\fi=0 % if pdftex
  \usepackage[T1]{fontenc}
  \usepackage[utf8]{inputenc}
\else % if luatex or xelatex
  \ifxetex
    \usepackage{mathspec}
  \else
    \usepackage{fontspec}
  \fi
  \defaultfontfeatures{Ligatures=TeX,Scale=MatchLowercase}
\fi
% use upquote if available, for straight quotes in verbatim environments
\IfFileExists{upquote.sty}{\usepackage{upquote}}{}
% use microtype if available
\IfFileExists{microtype.sty}{%
\usepackage{microtype}
\UseMicrotypeSet[protrusion]{basicmath} % disable protrusion for tt fonts
}{}
\usepackage[margin=1in]{geometry}
\usepackage{hyperref}
\hypersetup{unicode=true,
            pdftitle={Data Science \#4 - Max Greene},
            pdfborder={0 0 0},
            breaklinks=true}
\urlstyle{same}  % don't use monospace font for urls
\usepackage{graphicx,grffile}
\makeatletter
\def\maxwidth{\ifdim\Gin@nat@width>\linewidth\linewidth\else\Gin@nat@width\fi}
\def\maxheight{\ifdim\Gin@nat@height>\textheight\textheight\else\Gin@nat@height\fi}
\makeatother
% Scale images if necessary, so that they will not overflow the page
% margins by default, and it is still possible to overwrite the defaults
% using explicit options in \includegraphics[width, height, ...]{}
\setkeys{Gin}{width=\maxwidth,height=\maxheight,keepaspectratio}
\IfFileExists{parskip.sty}{%
\usepackage{parskip}
}{% else
\setlength{\parindent}{0pt}
\setlength{\parskip}{6pt plus 2pt minus 1pt}
}
\setlength{\emergencystretch}{3em}  % prevent overfull lines
\providecommand{\tightlist}{%
  \setlength{\itemsep}{0pt}\setlength{\parskip}{0pt}}
\setcounter{secnumdepth}{0}
% Redefines (sub)paragraphs to behave more like sections
\ifx\paragraph\undefined\else
\let\oldparagraph\paragraph
\renewcommand{\paragraph}[1]{\oldparagraph{#1}\mbox{}}
\fi
\ifx\subparagraph\undefined\else
\let\oldsubparagraph\subparagraph
\renewcommand{\subparagraph}[1]{\oldsubparagraph{#1}\mbox{}}
\fi

%%% Use protect on footnotes to avoid problems with footnotes in titles
\let\rmarkdownfootnote\footnote%
\def\footnote{\protect\rmarkdownfootnote}

%%% Change title format to be more compact
\usepackage{titling}

% Create subtitle command for use in maketitle
\newcommand{\subtitle}[1]{
  \posttitle{
    \begin{center}\large#1\end{center}
    }
}

\setlength{\droptitle}{-2em}

  \title{Data Science \#4 - Max Greene}
    \pretitle{\vspace{\droptitle}\centering\huge}
  \posttitle{\par}
    \author{}
    \preauthor{}\postauthor{}
    \date{}
    \predate{}\postdate{}
  

\begin{document}
\maketitle

\section{Question \#1}\label{question-1}

1(a): A function can be passed arguments as inputs and perform
computations on them. 1(b): Sys.Date returns a ``Date'' object
corrosponding to the ``system's idea'' of the date, according to the R
documentation. 1(c): 1. Everything that exists is an object. 2.
Everything that happens is a function call. 1(d): The source code for
any function can be seen by entering the function name into the console
without ant parentheses or arguments. 1(e): Default arguments can be
useful to save time and reduce function complexity while allowing for
function versatility. Specifically, they'er useful when you think the
user will enter the same value into the function most of the time. 1(f):
The \emph{args} function returns the arguments that the functions
accepts. For example, \texttt{args(mean)} returns
\texttt{function\ (x,...)} 1(g): Passing functions as arguments
increases the versatility of the function. For example,
\texttt{evaluate()} in swirl became a function capable of many
computations from modulus operations to means to rounding. 1(h):
\texttt{my\_vector{[}length(my\_vector){]}} -OR-
\texttt{tail(my\_vector,1)} 1(i): \texttt{paste()} combines all string
arguments into a single string, seperated by the \texttt{sep\ =\ "\ "}
argument. 1(j): the ``dot-dot-dot'' argument in R allows for an infinite
amount of arguments, which can be unpacked based on their assignment.

\section{Question \#2}\label{question-2}

\begin{verbatim}
 my_mean_func <- function(x)
 {
    return(mean(x))
 }
\end{verbatim}

\section{Question \#3}\label{question-3}

\begin{verbatim}
my_remain_func <- function(num, divisor)
{
  return(num %% divisor)
}
\end{verbatim}

\section{Question \#4}\label{question-4}

4(a): \emph{Scoring functions} are a form of data manipulation that
reduces multidimensional data into a single numerical value. 4(b):
According to 4.2, \emph{Scores} are functions that extract the features
of each entity and map them to a single numerical value. 4(c): An
effective scoring system should be easily computable, easily
understandable, intuitive (monotonic variable interpretation), produce
accurate results on outliers and utilize a normal distribution. 4(d): A
ranking is a certain type of score meant to list objects in order of the
score. For example, sports team rankings, University rankings, student
rankings on GPA, and Google PageRank are all ranking scores. 4(e): 4(f):
Z-scores are functions applied to a set of data in order to normalize
it. In a data set with non-zero mean and standard deviation, a Z-score
normalizes the mean to 0 and the standard deviation to 1, maintaining
the relative differences between object values.

\section{Question \#5}\label{question-5}

5(a):

\begin{verbatim}
my_OBP <- function(hits,walks,atBats)
{
  return((hits+walks)/atBats)
}
\end{verbatim}

5(b):

\begin{verbatim}
H <- sample(162:16200,270,replace=TRUE)
W <- sample(162:1782,270,replace=TRUE)
AB <- sample(64800:105300,270,replace=TRUE)

OBP <- my_OBP(H,W,AB)
\end{verbatim}

The \texttt{sample()} function takes a specified number of items of a
given vector, creating a somewhat random sampling of data from an evenly
distributed vector.

5(c):

\begin{verbatim}
mean(OBP)
\end{verbatim}

\section{Question \#6}\label{question-6}

My function \texttt{my\_VO2\_max} is used to determine the maximum
oxygen uptake of an individual. Here, I use it to estimate an average of
the population. The average female VO2 max tends to range from 27-30
mL/kg/min and the average male ranges from 35-40 mL/kg/min. It is
assumed that larger individuals will need to intake more oxygen to
support more cells, so by dividing my mass of the individual, we
normalize for mass and create an efficiency metric. VO2 max is used by
endurance athletes to determine the efficiency of their oxygen uptake.

\begin{verbatim}
my_VO2_max <- function(mLO2,duration,mass)
{
  return((mLO2/duration)/mass)
}

mLO2 <- sample(c(5000:100000),500,replace=TRUE)
duration <- sample(c(5:60),500,replace=TRUE)
mass <- sample(c(50:120),500,replace=TRUE)

avg_VO2_max <- mean(my_VO2_max(mLO2,duration,mass))
\end{verbatim}

This is a good scoring function because it is accurate and a real world
equation. If the data collected is accurate, this function will return
the amount of oxygen used in a certain amount of time, normalized by the
weight of the individual.

\section{Question \#7}\label{question-7}

\begin{verbatim}
library(nycflights13)

arr_delay_2h <- filter(flights, arr_delay > 120)

to_Houston <- filter(flights, dest == "IAH" | dest == "HOU")

United_American_Delta <- filter(flights, carrier == "UA" | carrier == "AA" | carrier == "DL" )

Summer <- filter(flights, month == 7 | month == 8 | month == 9)

arrived_late <- filter(flights, arr_delay > 120 & dep_delay < 1)

delayed_but_quick <- filter(flights, dep_delay > 60 & arr_delay < dep_delay-30)

early_morning <- filter(flights , dep_time >= 0000 & dep_time <= 0600)
\end{verbatim}

\section{Question \#8}\label{question-8}

\texttt{dplyr::between} is a shortcut for checking if a value is between
two numerical values. For example, if you are checking for flights from
midnight to 6:00am, you could have \texttt{between(dep\_time,0000,0600)}
instead of
\texttt{dep\_time\ \textgreater{}=\ 0000\ \&\ dep\_time\ \textless{}=\ 0600}.

\section{Question \#9}\label{question-9}

\begin{verbatim}
filter(flights, is.na(dep_time))
\end{verbatim}

There are 8,255 objects missing \texttt{dep\_time}. these objects are
also missing \texttt{dep\_delay}, \texttt{arr\_time},
\texttt{arr\_delay}, \texttt{air\_time}, and the rest of the data
collected from the flight actually taking place. All data scheduling the
flights are here. This may mean the fligths were scheduled bu cancelled,
possibly to inclement weather.

\section{Question \#10}\label{question-10}

\begin{verbatim}
arrange(flights, desc(is.na(dep_time)))
\end{verbatim}

\section{Question \#11}\label{question-11}

\begin{verbatim}
arrange(flights, desc(dep_delay))
\end{verbatim}

\section{Question \#12}\label{question-12}

\begin{verbatim}
arrange(flights, air_time)
\end{verbatim}


\end{document}
